We also compare our results to mock catalogues based on the methodology presented in \cite{Cowley14}, for full details we refer the reader to Cowley et al. but summarise it briefly here. An updated version of the \galform \citep[e.g.][Lacey et al. in preparation]{Cole00} semi-analytic galaxy formation model is used to populate halo merger trees \citep[e.g.][]{PCH08,Jiang14} derived from a Millennium style $N$-body dark matter only simulation \citep{Springel05,Guo13} with WMAP7 cosmology \citep{Komatsu11}.  Submm flux is calculated using a model based on the radiative transfer.  Dust is assumed to exist in two components,  dense molecular clouds and a diffuse ISM.  Energy absorbed from starlight by each dust component is calculated self-consistently. The dust is then assumed to emit radiation as a modified blackbody, assuming thermal equilibrium.  

Three randomly orientated 20 deg$^2$ lightcone catalogues are generated using the method described in \cite{Merson13}.  We choose as the lower flux limit for inclusion of simulated galaxies into our lightcone catalogue $S_{500\mu\rm{m}}>0.1$ mJy, as this is the limit at which we recover 90 per cent of the extragalactic background light (EBL) as predicted by our model (122 Jy deg$^{-2}$). This is in excellent agreement with observations from the \emph{COBE} satellite \citep[e.g.][]{Puget96,Fixsen98}. This ensures we have a realistic 500 $\mu$m  background in our mock images.  

Mock imaging is created by binning the lightcone galaxies onto a pixelated grid which is then convolved with a  36 arcsec FWHM Gaussian ($\sim$ \emph{Herschel} SPIRE beam at 500 $\mu$m).  The image is then constrained to have a zero mean by the subtraction of a uniform background.  For simplicity we do not attempt to add any instrumental noise, nor perform any further filtering of the mock image.  

This procedure is repeated at 350 and 250 $\mu$m in order to provide simulated \emph{Herschel} photometry.  We adjust the FWHM of the Gaussian PSF to 25 (18) arcsec at 350 (250) $\mu$m, and change the lower limit of inclusion into our lightcone to ensure 90 per cent of the predicted EBL is recovered at each wavelength.    

Source positions are then selected as maxima in the mock 250 $\mu$m image, with the position and flux of the source being recorded as the center and value of the maximal pixel.  To simulate `deblended' \emph{Herschel} photometry we simply read off the value of the pixel located at the position of the $250$ $\mu$m maxima in the 350 and 500 $\mu$m images. For all \emph{Herschel} sources with $S_{500\mu\rm{m}}>50$ mJy we then identify galaxies from our lightcone catalogues within a 9 arcsec radius ($\sim$ ALMA primary beam) of the source position, modelling the ALMA primary beam profile as a Gaussian with an 18 arcsec FWHM and a maximum sensitivity of 1 mJy.
